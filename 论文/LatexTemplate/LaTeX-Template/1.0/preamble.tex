%    This program is free software: you can redistribute it and/or modify
%    it under the terms of the GNU General Public License as published by
%    the Free Software Foundation, either version 3 of the License, or
%    (at your option) any later version.
%
%    This program is distributed in the hope that it will be useful,
%    but WITHOUT ANY WARRANTY; without even the implied warranty of
%    MERCHANTABILITY or FITNESS FOR A PARTICULAR PURPOSE.  See the
%    GNU General Public License for more details.
%
%    You should have received a copy of the GNU General Public License
%    along with this program.  If not, see <http://www.gnu.org/licenses/>.

%========================================
%	苏州大学论文LaTeX模板
%	2010年 05月 29日 星期六 19:47:22 CST
%	By telive
%	Email:	tellive@gmail.com
%========================================

%========================================
%		Packages used in this template
\usepackage[BoldFont, SlantFont]{xeCJK}			% 中文支持
\usepackage{pdfpages}		% 插入pdf
\usepackage{graphicx}		% 图形支持
\usepackage{epsfig}
\usepackage{titletoc}
\usepackage{subfigure, float}
\usepackage{indentfirst}
\usepackage{calc}
\usepackage{amssymb, amsmath}	% 数学符号
\usepackage{fancybox}		% 方框文字
\usepackage{wrapfig}		% 图片文字环绕
\usepackage{fancyhdr}		% 页眉设置
\usepackage{cite}			% 引用文献
\usepackage{indentfirst}	% 首行缩进
\usepackage[colorlinks,linkcolor=blue,citecolor=blue]{hyperref}			% 让 TOC 支持超链接
\usepackage[top=3.3cm,bottom=2.7cm,left=2.75cm,right=2.75cm]{geometry}  %设置页边距(学校的要求)
\usepackage{fontspec}
\usepackage{caption}
\usepackage{color}
\usepackage{array}
\newcommand{\PreserveBackslash}[1]{\let\temp=\\#1\let\\=\temp}
\newcolumntype{C}[1]{>{\PreserveBackslash\centering}p{#1}}
\newcolumntype{R}[1]{>{\PreserveBackslash\raggedleft}p{#1}}
\newcolumntype{L}[1]{>{\PreserveBackslash\raggedright}p{#1}}

%========================================
%		Settings
\setmainfont{Times New Roman}
\setCJKmainfont{SimSun}					% 中文默认字体设置为宋体
\setCJKfamilyfont{Heiti}{SimHei}		% Use \CJKfamily{Heiti} where you need.
\setCJKfamilyfont{Songti}{SimSun} 		% Use \CJKfamily{Songti} where you need.
\setCJKfamilyfont{Kaiti}{simkai.ttf} 	% Use \CJKfamily{Kaiti} where you need.
\setlength{\parindent}{2em}				% 首行缩进,2字符
\numberwithin{equation}{section}   		% 使公式标号为 3.1 的形式

\newcommand\Heiti{\CJKfamily{Heiti}}
\newcommand\Songti{\CJKfamily{Songti}}
\newcommand\Kaiti{\CJKfamily{Kaiti}}

%========================================
%		Redefine commands
\renewcommand\abstractname{\Large \bfseries \Songti 摘\ 要}	% 摘要
\renewcommand{\figurename}{\Songti 图} 					     % 图
\renewcommand{\tablename}{\Songti 表}           			     % 表
\renewcommand\refname{\Songti 参考文献}				 	      % 参考文献
\renewcommand\contentsname{\centerline{\Songti 目录}}	        % 目录居中
\renewcommand{\today}{\number\year 年 \number\month 月 \number\day 日}	%中文日期
%\renewcommand{\theequation}{\arabic{chapter}-\arabic{equation}}

%========================================
%		Header Settings
\pagestyle{fancy}

\titlecontents{section}
	[3cm]
	{\CJKfamily{Heiti}}
	{\contentslabel{2.5em}}
	{}
	{\titlerule*[0.5pc]{$\cdot$}\contentspage\hspace*{2cm}}
\titlecontents{subsection}
	[4cm]
	{\normalsize}
	{\contentslabel{2.5em}}
	{}
	{\titlerule*[0.5pc]{$\cdot$}\contentspage\hspace*{2cm}}
\titlecontents{subsubsection}
	[5cm]
	{\normalsize}
	{\contentslabel{2.5em}}
	{}
	{\titlerule*[0.5pc]{$\cdot$}\contentspage\hspace*{2cm}}

\chead{ \footnotesize  \Songti 苏州大学本科生毕业设计(论文)}	% 页眉中部
\lhead{}		% 页眉左部,设为空
\rhead{}		% 页眉右部,设为空

%========================================
%		Caption settings

\DeclareCaptionFont{kaiti}{\Kaiti}
\DeclareCaptionFont{bfheiti}{\bf\Heiti}
\captionsetup{font=small, format=plain, labelfont=bfheiti,
	textfont=kaiti, justification=raggedright,
	singlelinecheck=false
}