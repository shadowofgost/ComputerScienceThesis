\section{结论}
通过以上几个数值实验,我们证明了机器学习对于处理活性物质研究中巨量的数据是非常有效的。

1. 卷积神经网络是一种监督学习方法,适合于从注明了类别的数据中学习数据的结构特征,进而对数据进行分类和识别,适合于处理分类和识别等任务。其中,VGG16卷积神经网络是一种高效的图像识别和分类模型,对活性物质研究中生成的图像数据具有很强的学习能力。

2. 迁移学习可以有效地把一个领域的知识,迁移到另外一个领域,从而在新领域获得更好的学习效果。经过预训练的VGG16模型具有很强的迁移学习能力,只需针对特定领域稍微进行训练,就能以极高的性能完成该领域的识别任务。

3. t-SNE算法能够把高维特征空间中的数据集映射到二维和三维空间中的点集,并保持原数据点之间的邻居关系,是一种有效的高维数据可视化方法。

4. $k$-均值聚类算法对于处理原生数据具有强悍的能力,适合于对数据进行预加工。利用$k$均值聚类的$SSE-k$曲线,我们可以确定最佳聚类数目,从而识别一个数据集合中有几类数据。

5. 利用VGG16卷积神经网络提取图像特征,对特征之间的余弦相似度进行降序排序,可以有效地对活性物质图像进行检索。而直接计算图像之间的余弦相似度,对于活性物质图像的检索,是一种无效的方法。

6. 通过t-SNE降维来计算图像之间的距离,和直接将图像展平来计算图像间的欧氏距离,都可以有效地对活性物质图像进行检索。